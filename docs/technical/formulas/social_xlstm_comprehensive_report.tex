\documentclass[11pt,a4paper]{article}
\usepackage[utf8]{inputenc}
\usepackage{amsmath}
\usepackage{amsfonts}
\usepackage{amssymb}
\usepackage{graphicx}
\usepackage{geometry}
\usepackage{xeCJK}
\usepackage{algorithm}
\usepackage{algorithmic}
\usepackage{booktabs}
\usepackage{multirow}
\usepackage{subcaption}

\geometry{margin=1in}
\setCJKmainfont{AR PL UMing TW MBE}

\title{Social-xLSTM: Integrating Social Pooling with Extended Long Short-Term Memory for Traffic Flow Prediction}
\author{Social-xLSTM Research Team}
\date{\today}

\begin{document}

\maketitle

\begin{abstract}
本研究提出了 Social-xLSTM,一個結合 Social Pooling 機制與 Extended Long Short-Term Memory (xLSTM) 架構的創新交通流量預測模型。通過深入分析 Social LSTM 原理和 xLSTM 技術細節,我們設計了一個無需預定義拓撲結構的時空建模框架。該模型為每個車輛檢測器(VD)維持獨立的 xLSTM,並通過座標驅動的 Social Pooling 實現空間感知的隱藏狀態交互。實驗設計預期能夠改善傳統 LSTM 的過擬合問題,同時提供更好的長期時間依賴建模能力。

\textbf{關鍵詞}: 時空預測, Social LSTM, xLSTM, 交通流量預測, 深度學習
\end{abstract}

\section{Introduction}

交通流量預測作為智慧交通系統的核心技術,一直面臨著時間依賴性和空間相關性建模的雙重挑戰。傳統的 Long Short-Term Memory (LSTM) 網路在時間序列建模方面表現優異,但在記憶容量和長期依賴處理方面存在局限性。同時,多個檢測點之間的空間交互關係往往被忽略或建模不充分。

近年來,兩個重要的技術發展為解決這些問題提供了新的可能性:

\begin{enumerate}
\item \textbf{Social LSTM} \cite{alahi2016social}: 提出了通過 Social Pooling 機制實現多個序列間空間交互的方法,在人體軌跡預測中取得了顯著成效。

\item \textbf{xLSTM} \cite{beck2024xlstm}: 擴展了傳統 LSTM 架構,引入指數門控機制和矩陣記憶體,大幅提升了記憶容量和長期依賴建模能力。
\end{enumerate}

本研究的核心貢獻在於:首次將 Social Pooling 機制與 xLSTM 架構相結合,設計了 Social-xLSTM 模型,為交通流量預測提供了一個創新且理論基礎紮實的解決方案。

\section{Background and Related Work}

\subsection{Social LSTM: Architecture and Mathematical Formulation}

Social LSTM \cite{alahi2016social} 的核心創新在於解決多個相關序列之間的依賴關係問題。傳統 LSTM 無法捕捉多個相關序列間的依賴關係,因此 Social LSTM 提出了 Social Pooling 機制來連接空間上鄰近的 LSTM。

\subsubsection{核心架構原則}

Social LSTM 的正確理解基於以下關鍵點:

\begin{enumerate}
\item \textbf{個別 LSTM 設計}: 每個個體(行人或 VD)擁有自己的 LSTM 網路
\item \textbf{權重共享}: 所有 LSTM 使用相同的架構和參數
\item \textbf{隱藏狀態交互}: Social Pooling 作用於隱藏狀態,不是原始特徵
\item \textbf{分散式預測}: 每個個體為自己生成預測
\end{enumerate}

\subsubsection{Social Pooling 數學公式}

對於第 $i$ 個個體在時間 $t$,Social 隱藏狀態張量的構建公式為:

\begin{equation}
\mathbf{H}_t^i(m,n,:) = \sum_{j \in \mathcal{N}_i} \mathbf{1}_{mn}[x_j^t - x_i^t, y_j^t - y_i^t] \mathbf{h}_j^{t-1}
\end{equation}

其中:
\begin{align}
\mathbf{H}_t^i &: \text{第 } i \text{ 個個體在時間 } t \text{ 的 Social 隱藏狀態張量} \\
\mathbf{h}_j^{t-1} &: \text{鄰居 } j \text{ 在時間 } t-1 \text{ 的隱藏狀態} \\
\mathbf{1}_{mn}[x,y] &: \text{指示函數,檢查位置 } (x,y) \text{ 是否在網格 } (m,n) \text{ 中} \\
\mathcal{N}_i &: \text{第 } i \text{ 個個體的鄰居集合}
\end{align}

完整的前向傳播流程如下:

\begin{algorithm}
\caption{Social LSTM Forward Pass}
\begin{algorithmic}
\FOR{每個個體 $i$}
    \STATE $\mathbf{r}_t^i = \phi(\mathbf{x}_t^i; \mathbf{W}_r)$ \COMMENT{位置嵌入}
\ENDFOR
\FOR{每個個體 $i$}
    \STATE 構建 Social 隱藏狀態張量 $\mathbf{H}_t^i$
    \STATE $\mathbf{e}_t^i = \phi(\mathbf{H}_t^i; \mathbf{W}_e)$ \COMMENT{嵌入 Social 張量}
    \STATE $\mathbf{h}_t^i = \text{LSTM}(\mathbf{h}_{t-1}^i, [\mathbf{r}_t^i, \mathbf{e}_t^i]; \mathbf{W}_l)$
\ENDFOR
\FOR{每個個體 $i$}
    \STATE $\text{prediction}_i = \text{predict}(\mathbf{h}_t^i)$ \COMMENT{個別預測}
\ENDFOR
\end{algorithmic}
\end{algorithm}

\subsection{xLSTM: Extended Long Short-Term Memory}

xLSTM \cite{beck2024xlstm} 通過兩個主要創新擴展了傳統 LSTM:指數門控機制和修改的記憶體結構。xLSTM 包含兩種變體:sLSTM (scalar LSTM) 和 mLSTM (matrix LSTM)。

\subsubsection{sLSTM: Scalar LSTM with Exponential Gating}

sLSTM 引入了指數門控和適當的正規化技術:

\textbf{門控計算}:
\begin{align}
\tilde{i}_t &= \mathbf{W}_i \mathbf{e}_t + \mathbf{R}_i \mathbf{h}_{t-1} + \mathbf{b}_i \\
\tilde{f}_t &= \mathbf{W}_f \mathbf{e}_t + \mathbf{R}_f \mathbf{h}_{t-1} + \mathbf{b}_f \\
\tilde{z}_t &= \mathbf{W}_z \mathbf{e}_t + \mathbf{R}_z \mathbf{h}_{t-1} + \mathbf{b}_z \\
\tilde{o}_t &= \mathbf{W}_o \mathbf{e}_t + \mathbf{R}_o \mathbf{h}_{t-1} + \mathbf{b}_o
\end{align}

\textbf{指數門控}:
\begin{align}
i_t &= \exp(\tilde{i}_t) \\
f_t &= \sigma(\tilde{f}_t) \\
z_t &= \tanh(\tilde{z}_t) \\
o_t &= \sigma(\tilde{o}_t)
\end{align}

\textbf{細胞狀態和隱藏狀態}:
\begin{align}
c_t &= f_t \odot c_{t-1} + i_t \odot z_t \\
n_t &= f_t \odot n_{t-1} + i_t \\
\tilde{h}_t &= \frac{c_t}{n_t} \\
h_t &= o_t \odot \tilde{h}_t
\end{align}

其中 $n_t$ 是正規化狀態以防止溢出。

\subsubsection{mLSTM: Matrix LSTM with Covariance Update}

mLSTM 使用矩陣值記憶體以提高存儲容量:

\textbf{查詢、鍵、值計算}:
\begin{align}
\mathbf{q}_t &= \mathbf{W}_q \mathbf{e}_t \\
\mathbf{k}_t &= \mathbf{W}_k \mathbf{e}_t \\
\mathbf{v}_t &= \mathbf{W}_v \mathbf{e}_t
\end{align}

\textbf{矩陣記憶體更新}:
\begin{align}
\mathbf{C}_t &= \mathbf{f}_t \odot \mathbf{C}_{t-1} + \mathbf{i}_t \odot (\mathbf{v}_t \mathbf{k}_t^T) \\
\mathbf{n}_t &= \mathbf{f}_t \odot \mathbf{n}_{t-1} + \mathbf{i}_t \odot \mathbf{k}_t
\end{align}

其中 $\mathbf{C}_t \in \mathbb{R}^{d \times d}$ 是矩陣記憶體,$\mathbf{n}_t \in \mathbb{R}^d$ 是正規化向量。

\textbf{記憶體檢索}:
\begin{equation}
\mathbf{h}_t = \mathbf{o}_t \odot \frac{\mathbf{C}_t \mathbf{q}_t}{\max(\mathbf{n}_t^T \mathbf{q}_t, 1)}
\end{equation}

\subsubsection{混合 xLSTM 區塊堆疊}

一個 Social-xLSTM 區塊結合了 sLSTM 和 mLSTM 組件:

\begin{equation}
\text{Block}(\mathbf{x}) = \text{LayerNorm}(\mathbf{x} + \text{mLSTM}(\text{sLSTM}(\mathbf{x})))
\end{equation}

完整模型包含 $L$ 個堆疊區塊:
\begin{align}
\mathbf{h}^{(0)} &= \text{InputEmbedding}(\mathbf{e}_t) \\
\mathbf{h}^{(\ell)} &= \text{Block}_\ell(\mathbf{h}^{(\ell-1)}), \quad \ell = 1, \ldots, L \\
\mathbf{y}_t &= \text{OutputLayer}(\mathbf{h}^{(L)})
\end{align}

\section{Social-xLSTM: Methodology}

\subsection{Problem Formulation}

考慮一個包含 $N$ 個車輛檢測器(VDs)的交通網路,每個 VD $i$ 位於固定座標 $(x_i, y_i)$。在時間步 $t$,每個 VD 的輸入特徵向量定義為:

\begin{equation}
\mathbf{x}_i^t = [q_i^t, v_i^t, o_i^t]^T
\end{equation}

其中:
\begin{align}
q_i^t &= \text{VD } i \text{ 在時間 } t \text{ 的流量} \\
v_i^t &= \text{VD } i \text{ 在時間 } t \text{ 的速度} \\
o_i^t &= \text{VD } i \text{ 在時間 } t \text{ 的佔有率}
\end{align}

我們的目標是預測每個 VD 在下一個時間步 $t+1$ 的交通狀態:
\begin{equation}
\hat{\mathbf{y}}_i^{t+1} = [\hat{q}_i^{t+1}, \hat{v}_i^{t+1}, \hat{o}_i^{t+1}]^T
\end{equation}

\subsection{Social-xLSTM Architecture}

Social-xLSTM 架構包含四個主要組件:

\begin{enumerate}
\item \textbf{Individual xLSTM Processor}: 每個 VD 的獨立 xLSTM 處理
\item \textbf{Social Pooling Layer}: 座標驅動的空間特徵聚合
\item \textbf{State Fusion Layer}: 個體與社交特徵融合
\item \textbf{Output Layer}: 每個 VD 的個別預測
\end{enumerate}

\subsubsection{Individual xLSTM Processing}

每個 VD $i$ 使用共享的 xLSTM 架構進行獨立處理:

\begin{equation}
\mathbf{h}_i^{(\text{ind})} = \text{xLSTM}_{\text{shared}}(\mathbf{x}_i^{1:t})
\end{equation}

其中 $\text{xLSTM}_{\text{shared}}$ 是所有 VD 共享的 xLSTM 模型,確保參數效率和知識共享。

\subsubsection{Coordinate-Driven Social Pooling}

\textbf{鄰居發現}:
對於目標 VD $i$,我們基於歐幾里得距離找到其空間鄰居:

\begin{equation}
\mathcal{N}_i = \{j : d(\mathbf{c}_i, \mathbf{c}_j) \leq R, j \neq i\}
\end{equation}

其中 $\mathbf{c}_i = (x_i, y_i)$ 是 VD $i$ 的座標,$R$ 是空間半徑參數。

\textbf{空間網格構建}:
我們構建一個 $M \times N$ 的空間網格,中心位於目標 VD:

\begin{equation}
\mathbf{S}_i^t(m, n, :) = \sum_{j \in \mathcal{N}_i} w_{ij} \cdot \mathbf{1}_{mn}[\Delta x_{ij}, \Delta y_{ij}] \cdot \mathbf{h}_j^{(\text{ind})}
\end{equation}

其中:
\begin{align}
\Delta x_{ij} &= x_j - x_i, \quad \Delta y_{ij} = y_j - y_i \\
w_{ij} &= \exp\left(-\frac{d_{ij}}{R/3}\right) \quad \text{(距離權重)} \\
\mathbf{1}_{mn}[\cdot] &: \text{網格指示函數}
\end{align}

網格坐標計算為:
\begin{align}
m &= \left\lfloor \frac{\Delta x_{ij} + R_x}{2R_x / M} \right\rfloor \\
n &= \left\lfloor \frac{\Delta y_{ij} + R_y}{2R_y / N} \right\rfloor
\end{align}

\textbf{空間特徵投影}:
\begin{equation}
\mathbf{s}_i^t = \text{Proj}(\text{flatten}(\mathbf{S}_i^t))
\end{equation}

其中投影函數定義為:
\begin{equation}
\text{Proj}(\mathbf{x}) = \text{ReLU}(\mathbf{W}_2 \text{ReLU}(\mathbf{W}_1 \mathbf{x} + \mathbf{b}_1) + \mathbf{b}_2)
\end{equation}

\subsubsection{State Fusion}

我們使用多層融合策略結合個體和社交特徵:

\begin{equation}
\mathbf{h}_i^{(\text{fused})} = \text{LayerNorm}(\mathbf{W}_f [\mathbf{h}_i^{(\text{ind})}; \mathbf{s}_i^t] + \mathbf{b}_f)
\end{equation}

\subsubsection{Individual Prediction}

每個 VD 使用其獨立的輸出層生成預測:

\begin{equation}
\hat{\mathbf{y}}_i^{t+1} = \mathbf{W}_{\text{out}}^{(i)} \mathbf{h}_i^{(\text{fused})} + \mathbf{b}_{\text{out}}^{(i)}
\end{equation}

\subsection{Complete Algorithm}

\begin{algorithm}
\caption{Social-xLSTM Forward Pass}
\begin{algorithmic}
\STATE \textbf{Input:} VD features $\{\mathbf{x}_i^{1:t}\}_{i=1}^N$, coordinates $\{(x_i, y_i)\}_{i=1}^N$
\STATE \textbf{Output:} Predictions $\{\hat{\mathbf{y}}_i^{t+1}\}_{i=1}^N$

\FOR{each VD $i = 1, \ldots, N$}
    \STATE // Individual xLSTM Processing
    \STATE $\mathbf{h}_i^{(\text{ind})} = \text{xLSTM}_{\text{shared}}(\mathbf{x}_i^{1:t})$
\ENDFOR

\FOR{each VD $i = 1, \ldots, N$}
    \STATE // Social Pooling
    \STATE Find spatial neighbors $\mathcal{N}_i = \{j : d(\mathbf{c}_i, \mathbf{c}_j) \leq R\}$
    \STATE Construct spatial grid $\mathbf{S}_i^t$ of size $M \times N$
    \FOR{each neighbor $j \in \mathcal{N}_i$}
        \STATE Compute relative position $(\Delta x_{ij}, \Delta y_{ij})$
        \STATE Assign $w_{ij} \cdot \mathbf{h}_j^{(\text{ind})}$ to grid cell $(m, n)$
    \ENDFOR
    \STATE $\mathbf{s}_i^t = \text{Proj}(\text{flatten}(\mathbf{S}_i^t))$
\ENDFOR

\FOR{each VD $i = 1, \ldots, N$}
    \STATE // State Fusion and Prediction
    \STATE $\mathbf{h}_i^{(\text{fused})} = \text{LayerNorm}(\mathbf{W}_f [\mathbf{h}_i^{(\text{ind})}; \mathbf{s}_i^t] + \mathbf{b}_f)$
    \STATE $\hat{\mathbf{y}}_i^{t+1} = \mathbf{W}_{\text{out}}^{(i)} \mathbf{h}_i^{(\text{fused})} + \mathbf{b}_{\text{out}}^{(i)}$
\ENDFOR
\end{algorithmic}
\end{algorithm}

\section{Loss Function and Training}

\subsection{Multi-Objective Loss Function}

我們使用多目標損失函數結合不同的預測誤差度量:

\begin{equation}
\mathcal{L} = \alpha \mathcal{L}_{\text{MAE}} + \beta \mathcal{L}_{\text{MSE}} + \gamma \mathcal{L}_{\text{MAPE}}
\end{equation}

其中:
\begin{align}
\mathcal{L}_{\text{MAE}} &= \frac{1}{N} \sum_{i=1}^N |\hat{\mathbf{y}}_i - \mathbf{y}_i|_1 \\
\mathcal{L}_{\text{MSE}} &= \frac{1}{N} \sum_{i=1}^N \|\hat{\mathbf{y}}_i - \mathbf{y}_i\|_2^2 \\
\mathcal{L}_{\text{MAPE}} &= \frac{1}{N} \sum_{i=1}^N \left|\frac{\hat{\mathbf{y}}_i - \mathbf{y}_i}{\mathbf{y}_i}\right|_1
\end{align}

推薦的權重設置為 $(\alpha, \beta, \gamma) = (0.4, 0.4, 0.2)$。

\subsection{Training Strategy}

訓練過程採用端到端的方式,所有 VD 的損失同時優化:

\begin{equation}
\mathcal{L}_{\text{total}} = \sum_{i=1}^N \mathcal{L}_i
\end{equation}

其中 $\mathcal{L}_i$ 是 VD $i$ 的個別損失。

\section{Computational Complexity}

\subsection{Social Pooling Complexity}

對於每個 VD $i$ 有 $|\mathcal{N}_i|$ 個鄰居:
\begin{equation}
\mathcal{O}(\text{Social Pooling}) = \mathcal{O}(|\mathcal{N}_i| \cdot d_h + M \cdot N \cdot d_h)
\end{equation}

\subsection{xLSTM Complexity}

對於單個 xLSTM 區塊:
\begin{align}
\mathcal{O}(\text{sLSTM}) &= \mathcal{O}(d_h^2) \\
\mathcal{O}(\text{mLSTM}) &= \mathcal{O}(d_h^3)
\end{align}

\subsection{Total Model Complexity}

對於 $N$ 個 VD,$L$ 層,$T$ 個時間步:
\begin{equation}
\mathcal{O}(\text{Social-xLSTM}) = \mathcal{O}(N \cdot T \cdot L \cdot (|\overline{\mathcal{N}}| \cdot d_h + d_h^3))
\end{equation}

其中 $|\overline{\mathcal{N}}|$ 是每個 VD 的平均鄰居數。

\section{Theoretical Advantages}

\subsection{Compared to Traditional LSTM}

\begin{table}[h]
\centering
\begin{tabular}{|l|l|l|}
\hline
特性 & Traditional LSTM & Social-xLSTM \\
\hline
記憶機制 & 標量記憶體 & 標量+矩陣記憶體 \\
門控機制 & Sigmoid & 指數門控 \\
空間建模 & 無 & Social Pooling \\
長期依賴 & 有限 & 增強 \\
過擬合風險 & 較高 & 預期較低 \\
拓撲依賴 & N/A & 無需預定義 \\
\hline
\end{tabular}
\caption{Social-xLSTM 與傳統 LSTM 的比較}
\end{table}

\subsection{Key Innovation Points}

\begin{enumerate}
\item \textbf{首次將 xLSTM 應用於交通預測}: 探索最新的記憶網路技術在時空預測中的潛力

\item \textbf{創新的時空混合建模}: sLSTM 專門處理時間序列特徵,mLSTM 處理空間關係

\item \textbf{座標驅動的 Social Pooling}: 完全基於地理座標的空間建模,無需預定義拓撲

\item \textbf{個體化與社交化的平衡}: 每個 VD 保持獨立性的同時實現空間感知

\item \textbf{無拓撲依賴的可擴展架構}: 新增 VD 時無需重新設計網路結構
\end{enumerate}

\section{Expected Improvements}

基於理論分析,Social-xLSTM 預期在以下方面取得改善:

\subsection{Over-fitting Mitigation}

傳統 LSTM 基準顯示嚴重過擬合(驗證 $R^2 = -6$),xLSTM 的指數門控和更大記憶容量預期能:
\begin{itemize}
\item 改善訓練/驗證指標差距至 < 2倍
\item 實現正的驗證 $R^2$ 值
\item 更好的泛化能力
\end{itemize}

\subsection{Long-term Dependencies}

xLSTM 的矩陣記憶體和指數門控機制預期提供:
\begin{itemize}
\item 更好的長期時間序列建模
\item 增強的記憶容量
\item 更穩定的梯度傳播
\end{itemize}

\subsection{Spatial Modeling}

Social Pooling 機制預期實現:
\begin{itemize}
\item 自動發現空間關係
\item 無需預定義拓撲的建模能力
\item 可解釋的空間交互學習
\end{itemize}

\section{Implementation Considerations}

\subsection{Hyperparameters}

\begin{table}[h]
\centering
\begin{tabular}{|l|l|l|}
\hline
參數 & 符號 & 推薦值 \\
\hline
網格大小 & $M \times N$ & $8 \times 8$ \\
空間半徑 & $R$ & 25000m \\
隱藏維度 & $d_h$ & 128 \\
xLSTM 區塊數 & $L$ & 6 \\
sLSTM 位置 & - & [1, 3] \\
學習率 & $\alpha$ & 0.001 \\
損失權重 & $(\alpha, \beta, \gamma)$ & $(0.4, 0.4, 0.2)$ \\
\hline
\end{tabular}
\caption{Social-xLSTM 超參數}
\end{table}

\subsection{Technical Dependencies}

\begin{itemize}
\item \textbf{xlstm 庫}: 提供 sLSTM 和 mLSTM 實現
\item \textbf{PyTorch}: 深度學習框架
\item \textbf{CUDA}: GPU 加速支援(推薦)
\end{itemize}

\section{Conclusions and Future Work}

\subsection{Contributions}

本研究提出的 Social-xLSTM 架構具有以下主要貢獻:

\begin{enumerate}
\item \textbf{理論創新}: 首次結合 Social Pooling 與 xLSTM,為交通預測提供新的理論框架

\item \textbf{技術進步}: 設計了無需預定義拓撲的時空建模方法

\item \textbf{架構優化}: 通過混合 sLSTM/mLSTM 實現時空特徵的專門化處理

\item \textbf{實用價值}: 提供了可直接應用於實際交通系統的解決方案
\end{enumerate}

\subsection{Future Research Directions}

\begin{enumerate}
\item \textbf{性能優化}: 探索 TFLA 內核等優化技術以提升計算效率

\item \textbf{多尺度建模}: 研究不同時空尺度下的模型適應性

\item \textbf{實時應用}: 開發適用於實時交通預測的輕量化版本

\item \textbf{遷移學習}: 探索跨區域、跨城市的模型遷移能力

\item \textbf{因果推理}: 整合因果推理機制以提升模型可解釋性
\end{enumerate}

\subsection{Expected Impact}

Social-xLSTM 預期在以下領域產生重要影響:

\begin{itemize}
\item \textbf{學術研究}: 為時空預測領域提供新的研究範式
\item \textbf{工業應用}: 改善智慧交通系統的預測準確性
\item \textbf{技術發展}: 推動 xLSTM 在時空建模中的應用發展
\end{itemize}

\section*{Acknowledgments}

感謝所有為 Social-xLSTM 研究做出貢獻的團隊成員,以及提供技術支援的相關開源社群。

\begin{thebibliography}{9}

\bibitem{alahi2016social}
Alahi, A., Goel, K., Ramanathan, V., Robicquet, A., Fei-Fei, L., \& Savarese, S. (2016). 
Social LSTM: Human trajectory prediction in crowded spaces. 
In \textit{Proceedings of the IEEE conference on computer vision and pattern recognition} (pp. 961-971).

\bibitem{beck2024xlstm}
Beck, M., Pöppel, K., Spanring, M., Auer, A., Prudnikova, O., Kopp, M., ... \& Hochreiter, S. (2024). 
xLSTM: Extended Long Short-Term Memory. 
In \textit{Thirty-eighth Conference on Neural Information Processing Systems}.

\bibitem{hochreiter1997lstm}
Hochreiter, S., \& Schmidhuber, J. (1997). 
Long short-term memory. 
\textit{Neural computation}, 9(8), 1735-1780.

\end{thebibliography}

\end{document}