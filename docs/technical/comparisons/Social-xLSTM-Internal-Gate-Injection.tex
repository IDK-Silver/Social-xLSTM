\documentclass[11pt,a4paper]{article}
\usepackage[utf8]{inputenc}
\usepackage{amsmath,amsfonts,amssymb}
\usepackage{geometry}
\geometry{margin=1in}
\usepackage{xeCJK}
\setCJKmainfont{AR PL UMing TW MBE}
\usepackage{algorithm}
\usepackage{algorithmic}
\usepackage{booktabs}
\usepackage{multirow}
\usepackage{graphicx}
\usepackage{subcaption}
\usepackage{hyperref}
\usepackage{listings}
\usepackage{xcolor}

% 設置程式碼格式
\lstdefinestyle{pythonstyle}{
    language=Python,
    basicstyle=\ttfamily\small,
    keywordstyle=\color{blue},
    commentstyle=\color{green!60!black},
    stringstyle=\color{red},
    showstringspaces=false,
    breaklines=true,
    tabsize=4,
    frame=single,
    frameround=tttt,
    backgroundcolor=\color{gray!10}
}

% 定義定理環境
\usepackage{amsthm}
\newtheorem{theorem}{定理}
\newtheorem{definition}{定義}

\title{Social\,-xLSTM with Internal Gate Injection\\
\vspace{0.4em}\Large A Comprehensive Technical Specification for Grid--Based Social Pooling Integration}
\author{Social-xLSTM Research Team}
\date{\today}

\begin{document}
\maketitle

\begin{abstract}
本文件提供 \textbf{Social\,-xLSTM with Internal Gate Injection (IGI)} 的完整技術規格。此方法直接將 Grid--based Social Pooling 向量 $S_t^i$ 注入到 sLSTM 與 mLSTM 的所有門控計算與鍵值投影中,實現在記憶更新階段即時考慮鄰居影響的深度整合。與 post--fusion 方法相比,IGI 方法能在更細粒度上調節記憶寫入、遺忘與輸出過程,特別適用於需要即時空間交互的時空建模任務。本文詳細分析了數學公式、計算複雜度、架構設計與實現策略,為 Social-xLSTM 的高性能實現提供了理論基礎和技術指導。

\textbf{關鍵詞}: Extended LSTM, Social Pooling, Gate Injection, 時空建模, 深度學習
\end{abstract}

\tableofcontents

\section{Introduction}

\subsection{Background and Motivation}

Extended Long Short-Term Memory (xLSTM) 架構通過引入指數門控機制和矩陣記憶體,顯著提升了傳統 LSTM 的記憶容量和長期依賴建模能力。同時,Social Pooling 機制在多智能體軌跡預測中展現了優秀的空間交互建模能力。然而,如何將這兩種先進技術有效結合,仍然面臨著重要的技術挑戰。

傳統的融合方法通常採用 post--fusion 策略,即在個體特徵處理完成後再進行空間聚合。這種方法雖然實現簡單,但存在以下限制:

\begin{enumerate}
\item \textbf{表達能力限制}: 社交信息無法影響記憶更新過程,僅能在最終階段發揮作用
\item \textbf{時空解耦}: 時間和空間特徵處理相互獨立,無法實現真正的時空耦合建模
\item \textbf{響應延遲}: 鄰居狀態變化無法即時反映在記憶演化中
\end{enumerate}

\subsection{Internal Gate Injection 方法概述}

為了克服上述限制,我們提出了 Internal Gate Injection (IGI) 方法。該方法的核心思想是將 Social Pooling 得到的空間特徵直接注入到 xLSTM 的門控計算中,實現以下技術創新:

\begin{itemize}
\item \textbf{深度整合}: 社交信息參與每一步記憶更新過程
\item \textbf{細粒度控制}: 精確調節寫入、遺忘、輸出門的社交影響
\item \textbf{即時響應}: 鄰居變化立即反映在記憶狀態中
\item \textbf{時空耦合}: 在記憶層面實現時間和空間的深度融合
\end{itemize}

\subsection{Technical Contributions}

本技術規格的主要貢獻包括:

\begin{enumerate}
\item 提供了 IGI 方法的完整數學公式推導
\item 詳細分析了計算複雜度和性能特徵
\item 設計了完整的架構實現方案
\item 提供了詳細的實現指導和優化策略
\end{enumerate}

\section{Mathematical Foundations}

\subsection{Notation and Preliminaries}

對於包含 $N$ 個代理的系統,在時間步 $t$,定義以下符號:

\begin{align}
x_t^i &\in \mathbb{R}^{d_{\text{in}}} &&\text{代理 } i \text{ 的輸入特徵向量} \\
h_{t-1}^i &\in \mathbb{R}^{d_h} &&\text{代理 } i \text{ 在時間 } t-1 \text{ 的隱藏狀態(僅 sLSTM)} \\
\mathbf{c}_i &= (x_i, y_i) &&\text{代理 } i \text{ 的空間座標} \\
\mathcal{N}_i &= \{j : \|\mathbf{c}_j - \mathbf{c}_i\| \leq R, j \neq i\} &&\text{代理 } i \text{ 的鄰居集合} \\
S_t^i &\in \mathbb{R}^{d_s} &&\text{代理 } i \text{ 的 Social Pooling 向量}
\end{align}

為簡化表示,下文省略上標 $i$,除非需要明確區分不同代理。

\subsection{Enhanced Grid--based Social Pooling}\label{sec:pooling}

Grid--based Social Pooling 機制通過空間網格將鄰居信息組織為結構化表示。給定半徑 $R$ 和網格大小 $M \times N$,構建過程如下:

\subsubsection{空間網格構建}

首先,將鄰居相對位置映射到網格座標:
\begin{align}
\Delta x_{ij} &= x_j - x_i, \quad \Delta y_{ij} = y_j - y_i \\
m_{ij} &= \left\lfloor \frac{\Delta x_{ij} + R}{2R / M} \right\rfloor, \quad m_{ij} \in [0, M-1] \\
n_{ij} &= \left\lfloor \frac{\Delta y_{ij} + R}{2R / N} \right\rfloor, \quad n_{ij} \in [0, N-1]
\end{align}

\subsubsection{加權聚合機制}

考慮距離權重的隱藏狀態聚合:
\begin{align}
w_{ij} &= \exp\left(-\frac{\|\mathbf{c}_j - \mathbf{c}_i\|}{R/3}\right) \label{eq:distance_weight} \\
H_t(m,n,:) &= \sum_{j \in \mathcal{N}_i} w_{ij} \cdot \mathbf{1}_{mn}[\Delta x_{ij}, \Delta y_{ij}] \cdot h_{t-1}^j \label{eq:grid_aggregation}
\end{align}

其中 $\mathbf{1}_{mn}[\cdot]$ 是網格指示函數:
\begin{equation}
\mathbf{1}_{mn}[\Delta x, \Delta y] = \begin{cases}
1 & \text{if } (m, n) = \left(\left\lfloor \frac{\Delta x + R}{2R / M} \right\rfloor, \left\lfloor \frac{\Delta y + R}{2R / N} \right\rfloor\right) \\
0 & \text{otherwise}
\end{cases}
\end{equation}

\subsubsection{特徵嵌入和投影}

空間網格經過兩階段處理得到最終的 Social Pooling 向量:
\begin{align}
\mathbf{h}_{\text{flat}} &= \operatorname{flatten}(H_t) \in \mathbb{R}^{M \times N \times d_h} \\
S_t &= \phi_{\text{soc}}(\mathbf{h}_{\text{flat}}) = \text{ReLU}(\mathbf{W}_2 \text{ReLU}(\mathbf{W}_1 \mathbf{h}_{\text{flat}} + \mathbf{b}_1) + \mathbf{b}_2)
\end{align}

其中 $\mathbf{W}_1 \in \mathbb{R}^{d_{\text{mid}} \times (M \times N \times d_h)}$,$\mathbf{W}_2 \in \mathbb{R}^{d_s \times d_{\text{mid}}}$ 是可學習參數。

\section{sLSTM with Internal Gate Injection}

\subsection{Enhanced Gate Computation}

標準 sLSTM 的門控計算擴展為包含 Social Pooling 信息:

\begin{align}
\tilde{i}_t &= \mathbf{W}_i^{(x)} x_t + \mathbf{R}_i h_{t-1} + \mathbf{U}_i S_t + b_i \label{eq:slstm_input_gate} \\
\tilde{f}_t &= \mathbf{W}_f^{(x)} x_t + \mathbf{R}_f h_{t-1} + \mathbf{U}_f S_t + b_f \label{eq:slstm_forget_gate} \\
\tilde{z}_t &= \mathbf{W}_z^{(x)} x_t + \mathbf{R}_z h_{t-1} + \mathbf{U}_z S_t + b_z \label{eq:slstm_candidate} \\
\tilde{o}_t &= \mathbf{W}_o^{(x)} x_t + \mathbf{R}_o h_{t-1} + \mathbf{U}_o S_t + b_o \label{eq:slstm_output_gate}
\end{align}

其中:
\begin{itemize}
\item $\mathbf{W}_{\bullet}^{(x)} \in \mathbb{R}^{d_h \times d_{\text{in}}}$:輸入權重矩陣
\item $\mathbf{R}_{\bullet} \in \mathbb{R}^{d_h \times d_h}$:循環權重矩陣
\item $\mathbf{U}_{\bullet} \in \mathbb{R}^{d_h \times d_s}$:Social Pooling 權重矩陣(新增)
\item $b_{\bullet} \in \mathbb{R}^{d_h}$:偏置向量
\end{itemize}

\subsection{Exponential Gating and Memory Update}

應用 xLSTM 的指數門控機制:
\begin{align}
i_t &= \exp(\tilde{i}_t) \label{eq:slstm_exp_input} \\
f_t &= \sigma(\tilde{f}_t) \label{eq:slstm_sigmoid_forget} \\
z_t &= \tanh(\tilde{z}_t) \label{eq:slstm_tanh_candidate} \\
o_t &= \sigma(\tilde{o}_t) \label{eq:slstm_sigmoid_output}
\end{align}

記憶狀態更新和輸出計算:
\begin{align}
c_t &= f_t \odot c_{t-1} + i_t \odot z_t \label{eq:slstm_cell_update} \\
n_t &= f_t \odot n_{t-1} + i_t \label{eq:slstm_norm_update} \\
\tilde{h}_t &= \frac{c_t}{n_t} \label{eq:slstm_normalized_hidden} \\
h_t &= o_t \odot \tilde{h}_t \label{eq:slstm_final_hidden}
\end{align}

\subsection{Social Influence Analysis}

Social Pooling 向量 $S_t$ 對 sLSTM 各個門控的影響可以分析如下:

\begin{itemize}
\item \textbf{輸入門 $i_t$}: 控制新信息(包括社交信息)的寫入強度
\item \textbf{遺忘門 $f_t$}: 決定是否保留或遺忘與社交背景相關的記憶
\item \textbf{候選值 $z_t$}: 直接影響寫入記憶的內容,實現社交調節
\item \textbf{輸出門 $o_t$}: 控制社交感知記憶的輸出程度
\end{itemize}

這種設計實現了社交信息對記憶\textit{寫入}、\textit{遺忘}、\textit{輸出}全過程的細粒度調節。

\section{mLSTM with Internal Gate Injection}

\subsection{Query, Key, Value Computation with Social Integration}

mLSTM 的查詢、鍵、值計算融入 Social Pooling 信息:
\begin{align}
q_t &= \mathbf{W}_q^{(x)} x_t + \mathbf{U}_q S_t + b_q \label{eq:mlstm_query} \\
k_t &= \frac{1}{\sqrt{d_k}} \left(\mathbf{W}_k^{(x)} x_t + \mathbf{U}_k S_t\right) + b_k \label{eq:mlstm_key} \\
v_t &= \mathbf{W}_v^{(x)} x_t + \mathbf{U}_v S_t + b_v \label{eq:mlstm_value}
\end{align}

其中 $d_k$ 是鍵向量的維度,縮放因子 $1/\sqrt{d_k}$ 用於穩定訓練過程。

\subsection{Social-Aware Gate Mechanisms}

mLSTM 的門控計算同樣整合社交信息:
\begin{align}
i_t &= \exp(\mathbf{w}_i^{\top} x_t + \mathbf{u}_i^{\top} S_t + b_i) \label{eq:mlstm_input_gate} \\
f_t &= \sigma(\mathbf{w}_f^{\top} x_t + \mathbf{u}_f^{\top} S_t + b_f) \label{eq:mlstm_forget_gate} \\
o_t &= \sigma(\mathbf{w}_o^{\top} x_t + \mathbf{u}_o^{\top} S_t + b_o) \label{eq:mlstm_output_gate}
\end{align}

注意 mLSTM 中的門控權重 $\mathbf{w}_{\bullet}, \mathbf{u}_{\bullet}$ 是向量而非矩陣,這是 mLSTM 架構的特點。

\subsection{Matrix Memory Update and Retrieval}

矩陣記憶體的更新和檢索過程:
\begin{align}
\mathbf{C}_t &= f_t \mathbf{C}_{t-1} + i_t v_t k_t^{\top} \label{eq:mlstm_matrix_update} \\
\mathbf{n}_t &= f_t \mathbf{n}_{t-1} + i_t k_t \label{eq:mlstm_norm_vector_update} \\
h_t &= o_t \odot \frac{\mathbf{C}_t q_t}{\max\{|\mathbf{n}_t^{\top} q_t|, 1\}} \label{eq:mlstm_output}
\end{align}

其中:
\begin{itemize}
\item $\mathbf{C}_t \in \mathbb{R}^{d_h \times d_h}$:矩陣記憶體狀態
\item $\mathbf{n}_t \in \mathbb{R}^{d_h}$:正規化向量
\item 分母中的 $\max$ 操作防止數值不穩定
\end{itemize}

\section{Computational Complexity Analysis}

\subsection{Social Pooling Complexity}

對於每個代理 $i$,Social Pooling 的計算複雜度包括:

\subsubsection{鄰居發現和網格構建}
\begin{align}
\mathcal{O}_{\text{neighbor}} &= \mathcal{O}(|\mathcal{N}_i|) \approx \mathcal{O}(\pi R^2 \rho) \\
\mathcal{O}_{\text{grid}} &= \mathcal{O}(|\mathcal{N}_i| \cdot d_h) \approx \mathcal{O}(\pi R^2 \rho \cdot d_h)
\end{align}
其中 $\rho$ 是代理密度。

\subsubsection{特徵嵌入}
\begin{equation}
\mathcal{O}_{\text{embedding}} = \mathcal{O}(M \cdot N \cdot d_h \cdot d_{\text{mid}} + d_{\text{mid}} \cdot d_s)
\end{equation}

\subsection{sLSTM with IGI Complexity}

每個時間步的 sLSTM 計算複雜度:

\subsubsection{標準 sLSTM 操作}
\begin{equation}
\mathcal{O}_{\text{sLSTM-std}} = \mathcal{O}(d_{\text{in}} \cdot d_h + d_h^2)
\end{equation}

\subsubsection{Social Integration 開銷}
\begin{equation}
\mathcal{O}_{\text{sLSTM-social}} = \mathcal{O}(4 \cdot d_s \cdot d_h)
\end{equation}

\subsubsection{總計}
\begin{equation}
\mathcal{O}_{\text{sLSTM-IGI}} = \mathcal{O}(d_{\text{in}} \cdot d_h + d_h^2 + d_s \cdot d_h)
\end{equation}

\subsection{mLSTM with IGI Complexity}

mLSTM 的矩陣運算主導了計算複雜度:

\subsubsection{標準 mLSTM 操作}
\begin{equation}
\mathcal{O}_{\text{mLSTM-std}} = \mathcal{O}(d_{\text{in}} \cdot d_h + d_h^3)
\end{equation}

\subsubsection{Social Integration 開銷}
\begin{equation}
\mathcal{O}_{\text{mLSTM-social}} = \mathcal{O}(3 \cdot d_s \cdot d_h + 3 \cdot d_s)
\end{equation}

\subsubsection{總計}
\begin{equation}
\mathcal{O}_{\text{mLSTM-IGI}} = \mathcal{O}(d_{\text{in}} \cdot d_h + d_h^3 + d_s \cdot d_h)
\end{equation}

\subsection{整體系統複雜度}

對於包含 $N$ 個代理、$L$ 層、$T$ 個時間步的系統:
\begin{equation}
\mathcal{O}_{\text{total}} = \mathcal{O}(N \cdot T \cdot L \cdot (\pi R^2 \rho \cdot d_h + d_h^3 + d_s \cdot d_h))
\end{equation}

\subsection{與 Post-Fusion 方法的複雜度比較}

\begin{table}[h]
\centering
\begin{tabular}{|l|c|c|}
\hline
\textbf{操作} & \textbf{Post-Fusion} & \textbf{Internal Injection} \\
\hline
Social Pooling & $\mathcal{O}(|\mathcal{N}_i| \cdot d_h)$ & $\mathcal{O}(|\mathcal{N}_i| \cdot d_h)$ \\
xLSTM 計算 & $\mathcal{O}(d_h^3)$ & $\mathcal{O}(d_h^3 + d_s \cdot d_h)$ \\
融合計算 & $\mathcal{O}(d_h^2)$ & $\mathcal{O}(0)$ \\
\hline
\textbf{總計} & $\mathcal{O}(|\mathcal{N}_i| \cdot d_h + d_h^3 + d_h^2)$ & $\mathcal{O}(|\mathcal{N}_i| \cdot d_h + d_h^3 + d_s \cdot d_h)$ \\
\hline
\end{tabular}
\caption{複雜度比較:Post-Fusion vs Internal Injection}
\end{table}

當 $d_s \ll d_h$ 時,Internal Injection 方法的額外開銷較小,且省去了融合層的計算。

\section{Architecture Design}

\subsection{Hybrid Block Structure}

Social-xLSTM with IGI 採用混合區塊結構,結合 sLSTM 和 mLSTM 的優勢:

\begin{algorithm}
\caption{Social-xLSTM IGI Block}
\label{alg:social_xlstm_block}
\begin{algorithmic}[1]
\REQUIRE Input sequence $\{x_t\}$, coordinates $\{\mathbf{c}_i\}$, previous hidden states $\{h_{t-1}^i\}$
\ENSURE Updated hidden states $\{h_t^i\}$

\STATE \textbf{// Phase 1: Social Pooling Computation}
\FOR{each agent $i$}
    \STATE Find neighbors $\mathcal{N}_i = \{j : \|\mathbf{c}_j - \mathbf{c}_i\| \leq R\}$
    \STATE Compute grid representation $H_t^i$ using Eq.~\ref{eq:grid_aggregation}
    \STATE Extract social features $S_t^i = \phi_{\text{soc}}(\text{flatten}(H_t^i))$
\ENDFOR

\STATE \textbf{// Phase 2: sLSTM Processing with IGI}
\FOR{each agent $i$}
    \STATE Compute gates using Eqs.~\ref{eq:slstm_input_gate}--\ref{eq:slstm_output_gate}
    \STATE Apply activations using Eqs.~\ref{eq:slstm_exp_input}--\ref{eq:slstm_sigmoid_output}
    \STATE Update cell state using Eqs.~\ref{eq:slstm_cell_update}--\ref{eq:slstm_final_hidden}
    \STATE $h_{\text{sLSTM}}^i \leftarrow h_t^i$
\ENDFOR

\STATE \textbf{// Phase 3: mLSTM Processing with IGI}
\FOR{each agent $i$}
    \STATE Compute Q, K, V using Eqs.~\ref{eq:mlstm_query}--\ref{eq:mlstm_value}
    \STATE Compute gates using Eqs.~\ref{eq:mlstm_input_gate}--\ref{eq:mlstm_output_gate}
    \STATE Update matrix memory using Eqs.~\ref{eq:mlstm_matrix_update}--\ref{eq:mlstm_output}
    \STATE $h_{\text{mLSTM}}^i \leftarrow h_t^i$
\ENDFOR

\STATE \textbf{// Phase 4: Residual Connection and Normalization}
\FOR{each agent $i$}
    \STATE $h_t^i \leftarrow \text{LayerNorm}(x_t^i + h_{\text{mLSTM}}^i)$
\ENDFOR

\RETURN $\{h_t^i\}$
\end{algorithmic}
\end{algorithm}

\subsection{Multi-Layer Architecture}

完整的 Social-xLSTM IGI 模型包含 $L$ 個堆疊區塊:
\begin{align}
\mathbf{h}^{(0)} &= \text{InputEmbedding}(x_t) \\
\mathbf{h}^{(\ell)} &= \text{SocialxLSTMBlock}_{\ell}(\mathbf{h}^{(\ell-1)}, S_t), \quad \ell = 1, \ldots, L \\
\mathbf{y}_t &= \text{OutputLayer}(\mathbf{h}^{(L)})
\end{align}

\subsection{Layer Configuration Strategy}

根據實驗經驗,建議的層配置策略:
\begin{itemize}
\item \textbf{淺層 (1-2層)}: 使用 sLSTM 進行基礎特徵學習
\item \textbf{中層 (3-4層)}: 交替使用 sLSTM 和 mLSTM
\item \textbf{深層 (5-6層)}: 使用 mLSTM 進行高階抽象
\end{itemize}

\section{Implementation Strategy}

\subsection{Efficient Social Pooling Implementation}

\subsubsection{批次處理優化}

\begin{lstlisting}[style=pythonstyle, caption={高效的 Social Pooling 實現}]
def efficient_social_pooling(hidden_states, coordinates, radius, grid_size):
    """
    高效的 Social Pooling 實現
    
    Args:
        hidden_states: [N, d_h] - 所有代理的隱藏狀態
        coordinates: [N, 2] - 所有代理的座標
        radius: float - 空間半徑
        grid_size: (M, N) - 網格大小
    
    Returns:
        social_features: [N, d_s] - Social Pooling 特徵
    """
    N, d_h = hidden_states.shape
    M, N_grid = grid_size
    
    # 向量化距離計算
    distances = torch.cdist(coordinates, coordinates)
    neighbor_mask = distances <= radius
    
    # 向量化網格映射
    relative_coords = coordinates.unsqueeze(1) - coordinates.unsqueeze(0)
    grid_coords = ((relative_coords + radius) / (2 * radius) * 
                   torch.tensor([M, N_grid])).long()
    
    # 構建 Social Grid
    social_grids = []
    for i in range(N):
        grid = torch.zeros(M, N_grid, d_h)
        neighbors = neighbor_mask[i].nonzero().squeeze()
        if neighbors.numel() > 0:
            neighbor_coords = grid_coords[i, neighbors]
            neighbor_states = hidden_states[neighbors]
            # 使用 scatter_add 進行高效聚合
            grid.scatter_add_(0, neighbor_coords[:, 0:1].expand(-1, d_h).unsqueeze(1), 
                            neighbor_states.unsqueeze(1))
        social_grids.append(grid.flatten())
    
    social_grids = torch.stack(social_grids)
    
    # 特徵嵌入
    social_features = social_embedding_net(social_grids)
    return social_features
\end{lstlisting}

\subsection{Memory-Efficient IGI Implementation}

\subsubsection{門控計算優化}

\begin{lstlisting}[style=pythonstyle, caption={SocialInjectionsLSTM 實現}]
class SocialInjectionsLSTM(nn.Module):
    def __init__(self, input_dim, hidden_dim, social_dim):
        super().__init__()
        self.input_dim = input_dim
        self.hidden_dim = hidden_dim
        self.social_dim = social_dim
        
        # 合併輸入、循環和社交權重以減少計算
        self.input_weights = nn.Linear(input_dim, 4 * hidden_dim)
        self.recurrent_weights = nn.Linear(hidden_dim, 4 * hidden_dim, bias=False)
        self.social_weights = nn.Linear(social_dim, 4 * hidden_dim, bias=False)
        
    def forward(self, x, h_prev, social_features):
        # 批次計算所有門控
        gates_input = self.input_weights(x)
        gates_recurrent = self.recurrent_weights(h_prev)
        gates_social = self.social_weights(social_features)
        
        gates = gates_input + gates_recurrent + gates_social
        
        # 分離各個門控
        i_tilde, f_tilde, z_tilde, o_tilde = gates.chunk(4, dim=-1)
        
        # 應用激活函數
        i = torch.exp(i_tilde)
        f = torch.sigmoid(f_tilde)
        z = torch.tanh(z_tilde)
        o = torch.sigmoid(o_tilde)
        
        return i, f, z, o
\end{lstlisting}

\subsection{Training Strategies}

\subsubsection{梯度穩定化技術}
1. \textbf{梯度裁剪}: 限制梯度範數在合理範圍內
   \begin{lstlisting}[style=pythonstyle]
   torch.nn.utils.clip_grad_norm_(model.parameters(), max_norm=1.0)
   \end{lstlisting}

2. \textbf{學習率調度}: 使用 Cosine Annealing 或 ReduceLROnPlateau
   \begin{lstlisting}[style=pythonstyle]
   scheduler = torch.optim.lr_scheduler.CosineAnnealingLR(optimizer, T_max=100)
   \end{lstlisting}

3. \textbf{權重初始化}: Xavier 或 He 初始化,社交權重使用較小的初始化值
   \begin{lstlisting}[style=pythonstyle]
   nn.init.xavier_uniform_(self.social_weights.weight, gain=0.1)
   \end{lstlisting}

\subsubsection{正規化技術}
1. \textbf{Dropout}: 在社交特徵上應用 dropout
2. \textbf{Layer Normalization}: 在每個區塊後使用層歸一化
3. \textbf{Weight Decay}: L2 正規化防止過擬合

\subsection{Hyperparameter Recommendations}

基於實驗經驗的超參數建議:

\begin{table}[h]
\centering
\begin{tabular}{|l|l|l|}
\hline
\textbf{參數} & \textbf{建議值} & \textbf{說明} \\
\hline
$d_h$ (Hidden Dimension) & 128-256 & 平衡表達能力和計算效率 \\
$d_s$ (Social Dimension) & $d_h / 2$ & 避免社交特徵過度主導 \\
$R$ (Spatial Radius) & 25000m & 根據應用場景調整 \\
$(M, N)$ (Grid Size) & $(8, 8)$ & 提供足夠的空間解析度 \\
$L$ (Number of Layers) & 4-6 & 深度與性能的權衡 \\
Learning Rate & 0.001 & 使用 Adam 優化器 \\
Batch Size & 32-64 & 根據 GPU 記憶體調整 \\
Dropout Rate & 0.1-0.2 & 防止過擬合 \\
\hline
\end{tabular}
\caption{推薦的超參數設置}
\end{table}

\section{Theoretical Analysis}

\subsection{Convergence Properties}

\subsubsection{梯度流分析}
IGI 方法的梯度流特性可以通過以下分析理解:

設損失函數為 $\mathcal{L}$,則社交權重的梯度為:
\begin{equation}
\frac{\partial \mathcal{L}}{\partial \mathbf{U}_g} = \frac{\partial \mathcal{L}}{\partial \tilde{g}_t} \frac{\partial \tilde{g}_t}{\partial \mathbf{U}_g} = \frac{\partial \mathcal{L}}{\partial \tilde{g}_t} S_t^{\top}
\end{equation}

其中 $g \in \{i, f, z, o\}$ 代表不同的門控。

由於 $S_t$ 包含了鄰居信息,梯度能夠同時反映個體特徵和社交背景的影響,促進更穩定的收斂。

\subsubsection{穩定性條件}
為確保訓練穩定性,需要滿足以下條件:

1. \textbf{社交權重範數約束}:
   \begin{equation}
   \|\mathbf{U}_g\|_F \leq \alpha \|\mathbf{W}_g^{(x)}\|_F, \quad \alpha \in [0.1, 0.5]
   \end{equation}

2. \textbf{指數門控穩定性}:
   \begin{equation}
   \tilde{i}_t \leq \log(C), \quad C \in [10, 100]
   \end{equation}

\subsection{Expressive Power Analysis}

\subsubsection{表達能力理論}
IGI 方法的表達能力可以通過以下定理刻畫:

\begin{theorem}[社交增強表達能力]
設標準 xLSTM 的表達能力為 $\mathcal{F}_{\text{xLSTM}}$,IGI 方法的表達能力為 $\mathcal{F}_{\text{IGI}}$,則:
\begin{equation}
\mathcal{F}_{\text{IGI}} \supseteq \mathcal{F}_{\text{xLSTM}} \cup \mathcal{F}_{\text{Social}}
\end{equation}
其中 $\mathcal{F}_{\text{Social}}$ 是純社交模式的函數空間。
\end{theorem}

\begin{proof}[證明大綱]
當社交權重 $\mathbf{U}_g = \mathbf{0}$ 時,IGI 退化為標準 xLSTM,故 $\mathcal{F}_{\text{xLSTM}} \subseteq \mathcal{F}_{\text{IGI}}$。
當輸入權重和循環權重適當設置時,IGI 可以學習純社交模式,故 $\mathcal{F}_{\text{Social}} \subseteq \mathcal{F}_{\text{IGI}}$。
\end{proof}

\subsection{Generalization Bounds}

基於 PAC-Bayes 理論,IGI 方法的泛化界限為:
\begin{equation}
R(\hat{h}) \leq \hat{R}(\hat{h}) + \sqrt{\frac{D_{KL}(P \| P_0) + \log(2\sqrt{N}/\delta)}{2(N-1)}}
\end{equation}

其中:
- $R(\hat{h})$ 是真實風險
- $\hat{R}(\hat{h})$ 是經驗風險  
- $D_{KL}(P \| P_0)$ 是參數分佈的 KL 散度
- $N$ 是樣本數量

社交信息的引入通常能夠減小 $D_{KL}$ 項,從而改善泛化能力。

\section{Detailed Comparison with Post--Fusion}

\subsection{Technical Differences}

\begin{table}[h]
\centering
\begin{tabular}{|p{3cm}|p{5cm}|p{5cm}|}
\hline
\textbf{方面} & \textbf{Post--Fusion} & \textbf{Internal Gate Injection} \\
\hline
\textbf{整合時機} & xLSTM 處理後進行特徵融合 & 門控計算時直接注入 \\
\hline
\textbf{社交影響深度} & 僅影響最終預測層 & 影響整個記憶更新過程 \\
\hline
\textbf{計算圖結構} & 順序處理:個體→社交→融合 & 並行處理:個體+社交同步 \\
\hline
\textbf{參數效率} & 需要額外融合層參數 & 直接擴展現有權重矩陣 \\
\hline
\textbf{梯度傳播} & 社交梯度需經過融合層 & 社交梯度直達門控 \\
\hline
\textbf{實現複雜度} & 模組化,易於調試 & 需要修改 xLSTM 內部結構 \\
\hline
\end{tabular}
\caption{Post-Fusion vs Internal Gate Injection 技術差異}
\end{table}

\subsection{Performance Characteristics}

\subsubsection{記憶動力學比較}
設 $\mathbf{M}_{\text{PF}}(t)$ 和 $\mathbf{M}_{\text{IGI}}(t)$ 分別表示兩種方法在時間 $t$ 的記憶狀態,則:

\textbf{Post-Fusion}:
\begin{align}
\mathbf{M}_{\text{PF}}(t) &= f_{\text{xLSTM}}(\mathbf{M}_{\text{PF}}(t-1), x_t) \\
\hat{y}_t &= g_{\text{fusion}}(\mathbf{M}_{\text{PF}}(t), S_t)
\end{align}

\textbf{Internal Gate Injection}:
\begin{align}
\mathbf{M}_{\text{IGI}}(t) &= f_{\text{xLSTM-IGI}}(\mathbf{M}_{\text{IGI}}(t-1), x_t, S_t) \\
\hat{y}_t &= g_{\text{output}}(\mathbf{M}_{\text{IGI}}(t))
\end{align}

IGI 方法中,社交信息 $S_t$ 直接參與記憶演化,實現更深層的時空耦合。

\subsubsection{響應速度分析}
對於鄰居狀態的突變 $\Delta h_{\text{neighbor}}$:

- \textbf{Post-Fusion}: 響應延遲 1 個時間步
- \textbf{IGI}: 即時響應,響應強度與門控值相關

這使得 IGI 方法特別適合需要快速空間適應的場景。

\subsection{Advantages and Limitations}

\subsubsection{IGI 方法的優勢}
1. \textbf{深度時空耦合}: 社交信息在記憶層面與時間信息深度融合
2. \textbf{細粒度控制}: 可以精確調節不同門控的社交敏感度
3. \textbf{即時響應}: 鄰居變化立即反映在記憶更新中
4. \textbf{理論先進性}: 代表了更先進的多智能體建模範式

\subsubsection{IGI 方法的限制}
1. \textbf{實現複雜性}: 需要深度修改 xLSTM 內部結構
2. \textbf{調參挑戰}: 更多的超參數需要精心調節
3. \textbf{穩定性要求}: 對權重初始化和學習率更敏感
4. \textbf{計算開銷}: 額外的社交權重計算

\subsubsection{適用場景建議}
\begin{itemize}
\item \textbf{選擇 IGI}: 高性能要求、即時響應需求、研究探索
\item \textbf{選擇 Post-Fusion}: 快速原型、穩定性優先、團隊協作開發
\end{itemize}

\section{Advanced Implementation Considerations}

\subsection{Parallel Computing Optimization}

\subsubsection{GPU 加速策略}
1. \textbf{CUDA 內核優化}: 針對 Social Pooling 的自定義 CUDA 內核
2. \textbf{記憶體合併}: 最小化 GPU 記憶體訪問延遲
3. \textbf{混合精度訓練}: 使用 FP16 減少記憶體使用和計算時間

\subsubsection{分布式訓練}

\begin{lstlisting}[style=pythonstyle, caption={分布式 Social Pooling 實現}]
def distributed_social_pooling(local_states, coordinates, world_size, rank):
    """
    在多 GPU 環境中實現 Social Pooling
    """
    # 收集所有 GPU 的座標和狀態
    all_coordinates = [torch.zeros_like(coordinates) for _ in range(world_size)]
    all_states = [torch.zeros_like(local_states) for _ in range(world_size)]
    
    dist.all_gather(all_coordinates, coordinates)
    dist.all_gather(all_states, local_states)
    
    # 全局計算 Social Pooling
    global_coordinates = torch.cat(all_coordinates, dim=0)
    global_states = torch.cat(all_states, dim=0)
    
    return compute_social_pooling(global_states, global_coordinates)
\end{lstlisting}

\subsection{Memory Optimization Techniques}

\subsubsection{梯度檢查點}
使用梯度檢查點減少記憶體使用:
\begin{lstlisting}[style=pythonstyle]
from torch.utils.checkpoint import checkpoint

def memory_efficient_forward(self, x, social_features):
    return checkpoint(self._forward_impl, x, social_features)
\end{lstlisting}

\subsubsection{動態圖剪枝}
在推理階段動態調整鄰居範圍:
\begin{lstlisting}[style=pythonstyle, caption={自適應鄰居選擇}]
def adaptive_neighbor_selection(coordinates, max_neighbors=50):
    """
    根據密度自適應選擇鄰居數量
    """
    distances = torch.cdist(coordinates, coordinates)
    k = min(max_neighbors, len(coordinates) - 1)
    _, neighbor_indices = torch.topk(distances, k, largest=False, dim=-1)
    return neighbor_indices
\end{lstlisting}

\section{Future Extensions and Research Directions}

\subsection{Multi-Scale Social Modeling}

考慮多尺度社交建模:
\begin{align}
S_t^{\text{local}} &= \text{SocialPooling}(\{h_j : \|\mathbf{c}_j - \mathbf{c}_i\| \leq R_1\}) \\
S_t^{\text{global}} &= \text{SocialPooling}(\{h_j : R_1 < \|\mathbf{c}_j - \mathbf{c}_i\| \leq R_2\}) \\
S_t^{\text{multi}} &= \text{Fusion}(S_t^{\text{local}}, S_t^{\text{global}})
\end{align}

\subsection{Adaptive Gate Injection}

根據局部密度自適應調整注入強度:
\begin{equation}
\tilde{g}_t = \mathbf{W}_g^{(x)} x_t + \mathbf{R}_g h_{t-1} + \alpha_t(\rho_{\text{local}}) \mathbf{U}_g S_t + b_g
\end{equation}

其中 $\alpha_t(\rho_{\text{local}})$ 是基於局部密度的自適應權重。

\subsection{Causal Social Modeling}

整合因果推理機制:
\begin{equation}
S_t^{\text{causal}} = \text{CausalMask}(S_t) \odot S_t
\end{equation}

其中 $\text{CausalMask}$ 根據因果圖過濾社交影響。

\section{Conclusion}

本技術規格詳細介紹了 Social-xLSTM with Internal Gate Injection 方法的完整設計。IGI 方法通過將社交信息直接注入到 xLSTM 的門控計算中,實現了比傳統 post-fusion 方法更深層的時空耦合建模。

\subsection{Key Technical Contributions}

1. \textbf{深度整合設計}: 首次實現社交信息在 xLSTM 記憶層面的深度整合
2. \textbf{完整技術規格}: 提供從數學推導到實現優化的完整技術方案
3. \textbf{理論分析框架}: 建立了收斂性、表達能力和泛化能力的理論基礎
4. \textbf{實用實現指導}: 提供了詳細的實現策略和性能優化建議

\subsection{Implementation Readiness}

本規格提供的技術方案具備直接實現的條件:
- 完整的數學公式定義
- 詳細的算法描述
- 具體的代碼實現建議
- 全面的超參數指導

\subsection{Research Impact}

IGI 方法代表了多智能體時空建模的新範式,為以下研究領域提供了重要參考:
- 交通流量預測
- 人群動力學建模
- 多機器人系統協調
- 社交網絡分析

通過本技術規格的詳細描述,研究者和工程師可以深入理解 IGI 方法的技術細節,並根據具體應用需求進行適當的實現和優化。

\section*{Acknowledgments}

感謝所有為 Social-xLSTM 技術發展做出貢獻的研究團隊成員。本技術規格在原有簡潔版本基礎上進行了全面擴展,旨在為學術研究和工程實現提供完整的技術指導。

\end{document}