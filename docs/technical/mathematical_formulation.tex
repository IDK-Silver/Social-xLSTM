\documentclass[11pt,a4paper]{article}
\usepackage[utf8]{inputenc}
\usepackage{amsmath}
\usepackage{amsfonts}
\usepackage{amssymb}
\usepackage{graphicx}
\usepackage{geometry}
\usepackage{xeCJK}
\usepackage{algorithm}
\usepackage{algorithmic}

\geometry{margin=1in}
\setCJKmainfont{Noto Sans CJK TC}

\title{Social-xLSTM Mathematical Formulation}
\author{Social-xLSTM Research Team}
\date{\today}

\begin{document}

\maketitle

\section{Introduction}

This document provides the complete mathematical formulation for the Social-xLSTM model, which combines Social Pooling mechanisms with extended Long Short-Term Memory (xLSTM) architectures for traffic flow prediction without requiring explicit topology information.

\section{Problem Formulation}

\subsection{Input Definition}

For each vehicle detector (VD) node $i$ at time step $t$, the input feature vector is defined as:

\begin{equation}
\mathbf{x}_i^t = [q_i^t, p_i^t, s_i^t, \ell_i, \mathbf{t}_i]^T
\end{equation}

where:
\begin{align}
q_i^t &= \text{current traffic volume at node } i \text{ at time } t \\
p_i^t &= \text{current lane occupancy at node } i \text{ at time } t \\
s_i^t &= \text{current lane speed at node } i \text{ at time } t \\
\ell_i &= \text{number of lanes at node } i \text{ (static)} \\
\mathbf{t}_i &= \text{temporal features (hour, day, etc.)}
\end{align}

Each node $i$ has fixed spatial coordinates $(x_i, y_i)$ representing its geographical location.

\section{Social Pooling Mechanism}

\subsection{Coordinate-Driven Grid Construction}

For target node $i$ at time $t$, we construct a spatial grid of size $M \times N$ centered at its coordinates $(x_i^t, y_i^t)$:

\begin{equation}
\text{Grid}_{i}^t \in \mathbb{R}^{M \times N \times d_h}
\end{equation}

where $d_h$ is the hidden state dimension.

\subsection{Neighbor Assignment}

For each neighbor node $j \in \mathcal{N}_i$, we compute the relative displacement:

\begin{equation}
\Delta x_{ij}^t = x_j^t - x_i^t, \quad \Delta y_{ij}^t = y_j^t - y_i^t
\end{equation}

The grid coordinates $(m, n)$ for neighbor $j$ are determined by:

\begin{align}
m &= \lfloor \frac{\Delta x_{ij}^t + R_x}{2R_x / M} \rfloor \\
n &= \lfloor \frac{\Delta y_{ij}^t + R_y}{2R_y / N} \rfloor
\end{align}

where $R_x$ and $R_y$ are the grid radius parameters.

\subsection{Social Tensor Construction}

The social tensor $\mathbf{H}_i^t$ aggregates hidden states from neighbors:

\begin{equation}
\mathbf{H}_i^t(m, n, :) = \sum_{j \in \mathcal{N}_i} \mathbf{1}_{mn}[\Delta x_{ij}^t, \Delta y_{ij}^t] \cdot \mathbf{h}_j^{t-1}
\end{equation}

where $\mathbf{1}_{mn}[\cdot]$ is the indicator function for grid cell $(m,n)$, and $\mathbf{h}_j^{t-1}$ is the hidden state of neighbor $j$ from the previous time step.

\subsection{Embedding and Concatenation}

The social tensor is flattened and concatenated with node features:

\begin{equation}
\mathbf{e}_i^t = \phi([\mathbf{x}_i^t; \text{flatten}(\mathbf{H}_i^t)])
\end{equation}

where $\phi(\cdot)$ is a nonlinear embedding function (typically a fully connected layer).

\section{Extended LSTM (xLSTM) Formulation}

\subsection{Scalar-memory LSTM (sLSTM)}

The sLSTM extends traditional LSTM with exponential gating and normalization:

\subsubsection{Gate Computations}

\begin{align}
\tilde{i}_t &= \mathbf{W}_i \mathbf{e}_t + \mathbf{R}_i \mathbf{h}_{t-1} + \mathbf{b}_i \\
\tilde{f}_t &= \mathbf{W}_f \mathbf{e}_t + \mathbf{R}_f \mathbf{h}_{t-1} + \mathbf{b}_f \\
\tilde{z}_t &= \mathbf{W}_z \mathbf{e}_t + \mathbf{R}_z \mathbf{h}_{t-1} + \mathbf{b}_z \\
\tilde{o}_t &= \mathbf{W}_o \mathbf{e}_t + \mathbf{R}_o \mathbf{h}_{t-1} + \mathbf{b}_o
\end{align}

\subsubsection{Exponential Gating}

\begin{align}
i_t &= \exp(\tilde{i}_t) \\
f_t &= \sigma(\tilde{f}_t) \\
z_t &= \tanh(\tilde{z}_t) \\
o_t &= \sigma(\tilde{o}_t)
\end{align}

\subsubsection{Cell State and Hidden State}

\begin{align}
c_t &= f_t \odot c_{t-1} + i_t \odot z_t \\
n_t &= f_t \odot n_{t-1} + i_t \\
\tilde{h}_t &= \frac{c_t}{n_t} \\
h_t &= o_t \odot \tilde{h}_t
\end{align}

where $n_t$ is the normalizer state to prevent overflow.

\subsection{Matrix-memory LSTM (mLSTM)}

The mLSTM uses matrix-valued memory for higher capacity storage:

\subsubsection{Query, Key, Value Computation}

\begin{align}
\mathbf{q}_t &= \mathbf{W}_q \mathbf{e}_t \\
\mathbf{k}_t &= \mathbf{W}_k \mathbf{e}_t \\
\mathbf{v}_t &= \mathbf{W}_v \mathbf{e}_t
\end{align}

\subsubsection{Matrix Memory Update}

\begin{align}
\mathbf{C}_t &= \mathbf{f}_t \odot \mathbf{C}_{t-1} + \mathbf{i}_t \odot (\mathbf{v}_t \mathbf{k}_t^T) \\
\mathbf{n}_t &= \mathbf{f}_t \odot \mathbf{n}_{t-1} + \mathbf{i}_t \odot \mathbf{k}_t
\end{align}

where $\mathbf{C}_t \in \mathbb{R}^{d \times d}$ is the matrix memory and $\mathbf{n}_t \in \mathbb{R}^d$ is the normalization vector.

\subsubsection{Memory Retrieval}

\begin{equation}
\mathbf{h}_t = \mathbf{o}_t \odot \frac{\mathbf{C}_t \mathbf{q}_t}{\max(\mathbf{n}_t^T \mathbf{q}_t, 1)}
\end{equation}

\section{Hybrid xLSTM Block Stack}

\subsection{Block Architecture}

A Social-xLSTM block combines both sLSTM and mLSTM components:

\begin{equation}
\text{Block}(\mathbf{x}) = \text{LayerNorm}(\mathbf{x} + \text{mLSTM}(\text{sLSTM}(\mathbf{x})))
\end{equation}

\subsection{Multi-Block Stack}

The complete model consists of $L$ stacked blocks:

\begin{align}
\mathbf{h}^{(0)} &= \text{Input Embedding}(\mathbf{e}_t) \\
\mathbf{h}^{(\ell)} &= \text{Block}_\ell(\mathbf{h}^{(\ell-1)}), \quad \ell = 1, \ldots, L \\
\mathbf{y}_t &= \text{Output Layer}(\mathbf{h}^{(L)})
\end{align}

\section{Complete Social-xLSTM Algorithm}

\begin{algorithm}
\caption{Social-xLSTM Forward Pass}
\begin{algorithmic}
\STATE \textbf{Input:} Node features $\{\mathbf{x}_i^t\}_{i=1}^N$, coordinates $\{(x_i, y_i)\}_{i=1}^N$
\STATE \textbf{Output:} Predictions $\{\hat{\mathbf{y}}_i^{t+1}\}_{i=1}^N$

\FOR{each node $i = 1, \ldots, N$}
    \STATE // Social Pooling
    \STATE Construct grid $\text{Grid}_i^t$ of size $M \times N$
    \FOR{each neighbor $j \in \mathcal{N}_i$}
        \STATE Compute relative position $(\Delta x_{ij}^t, \Delta y_{ij}^t)$
        \STATE Assign $\mathbf{h}_j^{t-1}$ to grid cell $(m, n)$
    \ENDFOR
    \STATE $\mathbf{H}_i^t \leftarrow$ aggregate grid tensor
    \STATE $\mathbf{e}_i^t \leftarrow \phi([\mathbf{x}_i^t; \text{flatten}(\mathbf{H}_i^t)])$
    
    \STATE // xLSTM Processing
    \STATE $\mathbf{h}_i^t \leftarrow \text{xLSTM-Stack}(\mathbf{e}_i^t, \mathbf{h}_i^{t-1})$
    \STATE $\hat{\mathbf{y}}_i^{t+1} \leftarrow \text{Output-Layer}(\mathbf{h}_i^t)$
\ENDFOR
\end{algorithmic}
\end{algorithm}

\section{Loss Function}

For traffic flow prediction, we use a combination of regression losses:

\begin{equation}
\mathcal{L} = \alpha \mathcal{L}_{\text{MAE}} + \beta \mathcal{L}_{\text{MSE}} + \gamma \mathcal{L}_{\text{MAPE}}
\end{equation}

where:
\begin{align}
\mathcal{L}_{\text{MAE}} &= \frac{1}{N} \sum_{i=1}^N |\hat{y}_i - y_i| \\
\mathcal{L}_{\text{MSE}} &= \frac{1}{N} \sum_{i=1}^N (\hat{y}_i - y_i)^2 \\
\mathcal{L}_{\text{MAPE}} &= \frac{1}{N} \sum_{i=1}^N \left|\frac{\hat{y}_i - y_i}{y_i}\right|
\end{align}

\section{Computational Complexity}

\subsection{Social Pooling Complexity}

For each node $i$ with $|\mathcal{N}_i|$ neighbors:
\begin{equation}
\mathcal{O}(\text{Social Pooling}) = \mathcal{O}(|\mathcal{N}_i| \cdot d_h + M \cdot N \cdot d_h)
\end{equation}

\subsection{xLSTM Complexity}

For a single xLSTM block:
\begin{align}
\mathcal{O}(\text{sLSTM}) &= \mathcal{O}(d_h^2) \\
\mathcal{O}(\text{mLSTM}) &= \mathcal{O}(d_h^3)
\end{align}

\subsection{Total Model Complexity}

For $N$ nodes, $L$ layers, and $T$ time steps:
\begin{equation}
\mathcal{O}(\text{Social-xLSTM}) = \mathcal{O}(N \cdot T \cdot L \cdot (|\overline{\mathcal{N}}| \cdot d_h + d_h^3))
\end{equation}

where $|\overline{\mathcal{N}}|$ is the average number of neighbors per node.

\section{Hyperparameters}

\begin{table}[h]
\centering
\begin{tabular}{|l|l|l|}
\hline
Parameter & Symbol & Typical Value \\
\hline
Grid size & $M \times N$ & $8 \times 8$ \\
Grid radius & $R_x, R_y$ & 1000m \\
Hidden dimension & $d_h$ & 128 \\
Number of layers & $L$ & 4 \\
Learning rate & $\alpha$ & 0.001 \\
Loss weights & $(\alpha, \beta, \gamma)$ & $(0.4, 0.4, 0.2)$ \\
\hline
\end{tabular}
\caption{Social-xLSTM Hyperparameters}
\end{table}

\section{Conclusion}

The Social-xLSTM model provides a mathematically rigorous framework for traffic flow prediction without requiring explicit topology information. The combination of coordinate-driven social pooling and extended memory mechanisms enables effective modeling of spatial-temporal dependencies in traffic networks with irregular node distributions.

\end{document}